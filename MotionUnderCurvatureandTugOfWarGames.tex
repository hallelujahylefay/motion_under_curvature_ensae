\documentclass{article}
\usepackage[utf8]{inputenc}
\usepackage{amsmath, amsthm, amssymb}
\usepackage{graphicx}
\usepackage[french, ruled, vlined]{algorithm2e}
\newtheorem{theorem}{Théorème}[section]
\newtheorem{lemme}{Lemme}
\usepackage[frenchb]{babel}
\newtheorem{definition}{Definition}
\begin{document}
\title{Jeu de tir à la corde, temps de sortie d'un domaine et équations d'Hamilton-Jacobi-Bellman}
\author{LE FAY Yvann}
\date{Septembre 2021}
\newpage
\begin{abstract}
	Nous développons l'exemple d'un jeu de type $\min \max$ à deux joueurs, en temps et à pas spatial discrets. L'objectif du premier joueur est de minimiser son temps de sortie d'un domaine $\Omega\subset \mathbb{R}^2$, l'objectif du second joueur est de l'en empêcher en perturbant de manière déterministe sa trajectoire. 

	Nous proposons une stratégie optimale en temps discret se basant sur le principe de programmation dynamique de Bellman. Nous nous appuyons sur \textit{A deterministic-control-based approach to motion by Curvature} de Robert V. Kohn et Sylvia Serfaty pour interpréter la stratégie optimale lorsque le paramètre de normalisation spatiale et temporelle tend vers $0$ comme des solutions à des équations elliptiques dégénérées d'Hamilton-Jacobi-Bellman.
	
	Pour un développement plus complet, nous redirigeons le lecteur vers le papier de Robert V. Kohn et Sylvia Serfaty.
\end{abstract}
\section{Le jeu de Paul et Carole}
Considérons le jeu suivant. Soit $\Omega$ un compact de $\mathbb{R}^2$, $\varepsilon$ un réel strictement positif. Il y a deux joueurs, Paul et Carole. Paul commence avec la donnée initiale $x\in\Omega$, son but est d'atteindre la frontière $\partial\Omega$. Le but de Carole est de l'en empêcher ou au moins de le ralentir. Le jeu se joue en temps discret, à chaque tour,
\begin{enumerate}
	\item Paul choisit une direction, i.e la donnée d'un vecteur unitaire, $v\in\mathbb{R}^2$.
	\item Carole choisit si $b\in\{-1, 1\}$ et remplace $v$ par $bv$.
	\item Paul se déplace d'une longueur $\sqrt{2}\varepsilon$, se déplaçant de $x$ à $x+\sqrt{2}\varepsilon bv$
\end{enumerate}
Est-ce que Paul peut nécessairement atteindre le bord ? Existe-t-il une stratégie optimale pour Paul, i.e une stratégie qui minimise son temps de sortie sachant que Carole choisira à chaque tour, $b$ de manière optimale ?

\section{Le temps minimal de sortie}
On définit le temps minimal de sortie normalisé de Paul partant d'un point $x$ et en suivant une stratégie optimale par $U^\varepsilon(x) = k\varepsilon^2$ où $k$ est le nombre d'étapes nécessaires à sa sortie.

Si Paul commence le jeu à $x\in\mathbb{R}^2\backslash \Omega$, alors il est déjà sorti et $U^\varepsilon(x) = 0$. S'il commence à un point $x\in\Omega$ tel que $x+\varepsilon\sqrt{2}v$ et $x-\varepsilon\sqrt{2}v$ appartiennent à $\mathbb{R}^2\backslash\Omega$, pour un certain vecteur unitaire $v$, alors Paul sort en un temps minimal $U^\varepsilon(x) = \varepsilon^2$. Une caractérisation de tels points est la suivante. Considérons tous les segments de longueur $2\sqrt{2}\varepsilon$ dont les extrémités sont sur $\partial \Omega$ ou dans $\mathbb{R}^2\backslash\Omega$. Les milieux de ces segments définissent une courbe fermée $\gamma_1$, qui partitionne $\mathb{R}^2$ en deux régions. Les points $x\in\Omega$ à partir desquels Paul peut quitter en $\Omega$ en une étape sont ceux qui forment $\gamma_1$ et ceux dans la région non bornée de $\mathbb{R}^2\backslash\gamma_1$. Notons $\Omega_1$ le complémentaire de l'ensemble de ces points, c'est-à-dire l'ensemble des $x\in\Omega$ à partir desquels Paul a besoin d'au moins deux étapes pour quitter. En réitérant cette procédure, on obtient une famille de sous-domaines, $\Omega_1\subset\Omega_2\subset\ldots$ avec $\partial\Omega_j=\gamma_j$, tels que Paul a besoin d'au moins $j+1$ étapes pour quitter $\Omega_j$. On a alors $U^\varepsilon = (j+1)\varepsilon^2$ sur $\Omega_j\backslash\Omega_{j+1}$. Un lemme nous assurera que si $\Omega$ est borné alors Paul peut forcément gagner en au plus $C/\varepsilon^2$. Ainsi $\Omega_j$ est vide pour $j$ suffisamment grand et le temps normalisé de sortie $U^\varepsilon$ est uniformément borné. La définition formelle de $U^\varepsilon$ est la suivante
\begin{definition}
	Soit $\Omega$ un domaine du plan, convexe ou non. Si $x\in\mathbb{R}^2\backslash\Omega$ alors $U^\varepsilon(x) = 0$. Pour $x\in\Omega$, $U^\varepsilon(x) = \varepsilon^2$ s'il existe un vecteur unitaire $v$ tel que $x+\varepsilon\sqrt{2}v$ et $x-\varepsilon\sqrt{2}v$ appartient à $\mathbb{R}^2\backslash \Omega$. Pour tout $k\geq 1$ et $x\in \Omega$, $U^\varepsilon(x) = k\varepsilon^2$ s'il existe un vecteur unitaire $v$ tel que $\max(U^\varepsilon(x+\varepsilon \sqrt{2}v), U^\varepsilon(x-\varepsilon \sqrt{2}v))=(k-1)\varepsilon^2$. On a le programme dynamique suivant
	\begin{align*}
		U^\varepsilon(x) =
		\left\{\begin{array}{ll}
	\min_{||v||=1}\max_{b=\pm 1} \{\varepsilon^2 + U^\varepsilon(x+\varepsilon\sqrt{2}bv)\} & \textup{ pour } x\in \Omega\\
 0 & \textup{ pour } x\in\mathbb{R}^2\backslash\Omega
\end{array}
\right.\ll
\end{align*}
\end{definition}
\subsection{Principe de programmation dynamique et majoration uniforme du temps de sortie}
La majoration du temps de sortie donnée par le lemme suivant nous permet d'implémenter le principe de programmation dynamique précédent.

\begin{lemme}
	Pour tout domaine planaire $\Omega$, $U^\varepsilon$ est uniformément borné par $U^\varepsilon \leq \textup{diam}(\Omega)^2/2$. Dans le cas où $\Omega$ est convexe, $U^\varepsilon \leq |\Omega|/2\pi$.
\end{lemme}





%Le temps de sortie (après normalisation) de Paul partant du point $x$ selon la stratégie optimale est notée $U^\varepsilon(x)$, si Paul met $k$ étapes à sortir de $\Omega$ en suivant la stratégie optimale, alors on pose $U^\varepsilon(x) = k\varepsilon^2$. Le principe de programmation dynamique assure que
%\begin{align*}
%U^{\varepsilon}(x) = \min_{||v||=1}\max_{b=\pm 1} \{\epsilon^2 + U^{\varepsilon}(x+\sqrt{2}\epsilon b v)\}
%\end{align*}


Nous proposons un algorithme de calcul de $U^\varepsilon(x)$ fondé sur le principe de programmation dynamique, en se restreignant à un sous-ensemble $V$ fini de vecteurs normés. La notation . est mise pour la concaténation. Aussi, $S_v(s)$ est mis pour la projection de $s\in S$ sur la coordonnée correspondant à l'ensemble $\mathcal{V}$, $S_v(s)[-1]$ est mis pour le dernier vecteur $v$ à avoir été ajouté à $\mathcal{V}$. Par abus de notation, on omet l'argument $s$, de manière à ce que $\{S_t : (S_v[-1], S_b[-1])=(v,b)\}$ dénote l'ensemble des valeurs de $t$ prises par les éléments $s\in S$ tels que pour tout élément $s$, le couple formé de la dernière coordonnée $v$ de $\mathcal{V}$ ($S_v(s)[-1]$) et la dernière coordonnée $b$ de $\mathcal{B}$ ($S_b(s)[-1]$) coïncident avec $(v,b)$.

\begin{algorithm}[H]
	$f : (x, t, \mathcal{V}, \mathcal{B})$\\
	\Si{$x\in \Omega$ \textup{ et } $t\leq \frac{\textup{diam}(\Omega)^2}{2}$}{
	$S \leftarrow \varnothing$\\
	\Pour{$(v,b)\in V\times\{-1,1\}$}{
		$S\leftarrow S . \{f(x+\sqrt{2}\varepsilon b v, t+\epsilon^2, \mathcal{V}.\{v\}, \mathcal{B}.\{b\})\}$}
$v^* \leftarrow \textup{argmin}_{v\in V}\{\max_{b\in\{-1, 1\}}\{S_t : (S_v[-1], S_b[-1])=(v, b)\}\}$\\
			$b^* \leftarrow \textup{argmax}_{b\in \{-1, 1\}} \{S_t : S_v[-1] = v^*\}$\\
			$\mathcal{V} \leftarrow \mathcal{V}.\{v^*\}$\\
			$\mathcal{B}\leftarrow \mathcal{B}.\{b^*\}$\\
			$t\leftarrow S_t : (S_v[-1], S_b[-1]) = (v^*, b^*)$\\
			$x\leftarrow x+\sqrt{2}\varepsilon b^* v^*$\\
		}
	\textup{Renvoyer } $(x,t,\mathcal{V},\mathcal{B})$
\end{algorithm}

L'algorithme termine. En effet, à chaque étape, l'argument en $t$ croit de $\varepsilon^2$ et est borné par $\textup{diam}(\Omega)^2/2$. A chaque appel de $f$, il y a $2|V|$ appels récursifs à $f$, le calcul de $v^*$ est linéaire en $|V|$, le reste se fait en temps constant. La complexité est de la forme $C(n) = 2|V|C(n+1) + 2|V|$ pour $n\leq \textup{diam}(\Omega)^2/2\varepsilon^2$. Ainsi pour $\varepsilon$ petit, on a une complexité en $O((2|V|)^{\textup{diam}(\Omega)^2/2\varepsilon^2})$.

Dans le cas convexe, la borne plus fine pour $t$ donne une complexité en $O((2|V|)^{|\Omega|/2\pi\varepsilon^2})$.

\subsection{Mouvement par courbure et EDP}
\subsubsection{Le cas convexe}
\subsubsection{Le cas non convexe}

\end{document}
